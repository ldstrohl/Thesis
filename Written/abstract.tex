% You insert your abstract in the space below.


%This document is a summary of some relevant commands needed to create
%a Master's thesis for the Department of Mathematics and Statistics
%using \LaTeX. Included are examples of equations, figures, tables, and
%theorems. The formats listed in this document have been approved by
%the Department of Mathematical Sciences and the Graduate Division and
%Research.  If you have any difficulties with any of the driver or
%style files, please see your graduate adviser.

%This is my abstract which describes my thesis. It isn't done yet; this is a placeholder.
%
%Many extraterrestrial missions require a powered descent phase. Because this phase is late in the mission its fuel efficiency has an outsized effect on payload capacity. This thesis presents a strategy for optimizing fuel use using well tested guidance algorithms and a unique strategy that reduces fuel consumption over conventional strategies by a significant margin and has wide applicability. 

Powered descent guidance is a critical part of any aerospace mission with a heavy payload in sparse atmosphere or vacuum. It is also finding use in terrestrial landing for vehicle recovery. Powered descent presents the challenge of landing a vehicle softly while using as little fuel as possible. While much work has been done on powered descent guidance and fuel optimality, little consideration has been made of ignition timing and powered descent initiation. The powered descent phase is typically specified by a pre-determined condition, altitude, or time.

Adaptive Powered Descent Initiation (PDI) is proposed and tested as a strategy of initiating powered descent guidance automatically for robustness and improved propellant performance. It effectiveness is tested by Monte Carlo simulation and results presented for a manned Mars mission.

PDI is found to be especially effective in atmosphere as it extends the aerobraking phase without sacrificing safety and gaining propellant efficiency. Its closed-loop formulation ensures reliability in dispersed conditions reflective of practical mission conditions for Mars landing.
